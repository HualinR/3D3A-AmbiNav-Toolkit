\documentclass[11pt, oneside]{article}

%   ==============================================================================
%   This file is part of the 3D3A AmbiNav Toolkit.
%   
%   Joseph G. Tylka <josephgt@princeton.edu>
%   3D Audio and Applied Acoustics (3D3A) Laboratory
%   Princeton University, Princeton, New Jersey 08544, USA
%   
%   MIT License
%   
%   Copyright (c) 2018 Princeton University
%   
%   Permission is hereby granted, free of charge, to any person obtaining a copy
%   of this software and associated documentation files (the "Software"), to deal
%   in the Software without restriction, including without limitation the rights
%   to use, copy, modify, merge, publish, distribute, sublicense, and/or sell
%   copies of the Software, and to permit persons to whom the Software is
%   furnished to do so, subject to the following conditions:
%   
%   The above copyright notice and this permission notice shall be included in all
%   copies or substantial portions of the Software.
%   
%   THE SOFTWARE IS PROVIDED "AS IS", WITHOUT WARRANTY OF ANY KIND, EXPRESS OR
%   IMPLIED, INCLUDING BUT NOT LIMITED TO THE WARRANTIES OF MERCHANTABILITY,
%   FITNESS FOR A PARTICULAR PURPOSE AND NONINFRINGEMENT. IN NO EVENT SHALL THE
%   AUTHORS OR COPYRIGHT HOLDERS BE LIABLE FOR ANY CLAIM, DAMAGES OR OTHER
%   LIABILITY, WHETHER IN AN ACTION OF CONTRACT, TORT OR OTHERWISE, ARISING FROM,
%   OUT OF OR IN CONNECTION WITH THE SOFTWARE OR THE USE OR OTHER DEALINGS IN THE
%   SOFTWARE.
%   ==============================================================================

% Required packages
\usepackage[letterpaper, margin=1in, includeheadfoot]{geometry}
\usepackage{hyperref}
\usepackage{tabularx}

% Header and Footer
\usepackage{fancyhdr}
\pagestyle{fancy}
\renewcommand{\headrulewidth}{1pt}
\lhead{}\chead{\textsc{3D Audio and Applied Acoustics Laboratory $\cdot$ Princeton University}}\rhead{}
\lfoot{}\cfoot{\thepage}\rfoot{}

\renewcommand{\abstractname}{Summary} % Activate to modify the name of the Abstract
%\renewcommand*{\thefootnote}{\fnsymbol{footnote}} % Activate to use symbols rather than numbers for footnotes

% Additional packages
\usepackage{graphicx}
\usepackage{amssymb}
\usepackage{amsmath}
\usepackage[numbers,square,sort]{natbib}
\usepackage{tikz}

% User-defined commands
\newcommand{\citeref}[1]{Ref.~\cite{#1}}
\newcommand{\Citeref}[1]{Reference~\cite{#1}}
\newcommand{\citerefs}[1]{Refs.~\cite{#1}}
\newcommand{\Citerefs}[1]{References~\cite{#1}}
\newcommand{\figref}[1]{Fig.~\ref{#1}}
\newcommand{\Figref}[1]{Figure~\ref{#1}}
\newcommand{\figreftwo}[2]{Figs.~\ref{#1} and~\ref{#2}}
\newcommand{\eqnref}[1]{Eq.~(\ref{#1})}
\newcommand{\eqnreftwo}[2]{Eqs.~(\ref{#1}) and~(\ref{#2})}
\newcommand{\secref}[1]{Section~\ref{#1}}
\newcommand{\Secref}[1]{Section~\ref{#1}}
\newcommand{\secreftwo}[2]{Sections~\ref{#1} and~\ref{#2}}
\newcommand{\secrefthru}[2]{Sections~\ref{#1} through~\ref{#2}}
\newcommand{\tabref}[1]{Table~\ref{#1}}
\newcommand{\Tabref}[1]{Table~\ref{#1}}
\newcommand{\tabreftwo}[2]{Tables~\ref{#1} and~\ref{#2}}

\begin{document}

% Title and Author block
\begin{centering}
{\Large \textbf{Ambisonics Navigation (AmbiNav) Toolkit}}\\
\vspace{\baselineskip}
Joseph G.~Tylka\\
\href{mailto:josephgt@princeton.edu}{josephgt@princeton.edu}\\
\vspace{\baselineskip}
v0 -- October 20\textsuperscript{th}, 2018\\
\end{centering}

% Abstract
\begin{abstract}
The ambisonics navigation (AmbiNav) toolkit is an open-source collection of MATLAB functions for performing virtual navigation of higher-order ambisonics recordings.
This document describes the methods implemented in the toolkit and provides instructions for its use.
\end{abstract}

\section{Introduction}
This document is structured as follows.
In~\secref{sec:Conventions}, we describe the mathematical conventions followed in this toolkit.
Finally, in~\secref{sec:Functions}, we describe the various functions included in the toolkit and how to use them.

\subsection{Contents}
The toolkit consists of the following items:
\begin{enumerate}
\item the core set of MATLAB functions for the toolkit (\texttt{/AmbiNav\_*.m});
\item the source \LaTeX~files for this manual (\texttt{/doc/});
\item a MATLAB file containing examples of how to use the toolkit (\texttt{/examples.m}); and
\item MATLAB implementations of some published virtual navigation methods (\texttt{/methods/}).
\end{enumerate}

%%%% Conventions %%%%
\section{Conventions}\label{sec:Conventions}
Here, we use real-valued spherical harmonics as given by~\citet[section~2.2]{Zotter2009PhD}
\begin{equation*}
Y_l^m(\theta,\phi) = N_l^{|m|} P_l^{|m|} (\sin \theta) \times \left\{
    \begin{array}{cl}
	\cos m \phi & \textrm{for } m \geq 0,\\[8pt]
	\sin |m| \phi & \textrm{for } m < 0,
    \end{array}\right.
\end{equation*}
where $P_l^m$ is the associated Legendre polynomial of degree $l$ and order $m$,
as defined in the MATLAB \texttt{legendre} function by\footnote{See:~\url{https://www.mathworks.com/help/matlab/ref/legendre.html}}
\begin{equation*}
P_l^m(x) = (-1)^m (1 - x^2)^{m/2} \frac{d^m}{dx^m} P_l(x), 
\quad\quad \textrm{with} \quad\quad
P_l(x) = \frac{1}{2^l l!} \left[ \frac{d^l}{dx^l}(x^2 - 1)^l \right],
\end{equation*}
and $N_l^m$ is a normalization term which, for the orthonormal (N3D)
spherical harmonics with Condon-Shortley phase,\footnote{Note that including
Condon-Shortley phase in the normalization term cancels it in the associated Legendre term.}
is given by~\citep{Nachbar2011}
\begin{equation*}
N_l^m = (-1)^m \sqrt{\frac{(2l+1)(2 - \delta_m)}{4 \pi} \frac{(l-m)!}{(l+m)!}},
\end{equation*}
where $\delta_m$ is the Kronecker delta.
With an inner product defined by integrating over all directions, the squared-norm of these spherical harmonics is thus given by
\begin{equation*}
\left\| Y_l^m \right\|^2 = 1.
\end{equation*}

We also adopt the ambisonics channel numbering (ACN) convention~\citep{Nachbar2011}
such that for a spherical harmonic function of degree $l \in [0,\infty)$ and order $m \in [-l,l]$,
the ACN index $n$ is given by $n = l (l + 1) + m$ and we denote the spherical harmonic function by $Y_n \equiv Y_l^m$.

\section{MATLAB Functions}\label{sec:Functions}
The core MATLAB functions included in the toolkit are listed and described in~\tabref{tab:Functions}.

\begin{table}
\centering
  \begin{tabular}{| l | p{11cm} |}
    \hline
    \textbf{Function} & \textbf{Description} \\ \hline
    \texttt{AmbiNav\_ArraySpacing} & Computes the (approximate) spacing between HOA microphones. \\ \hline
    \texttt{AmbiNav\_CoefficientA} & Returns the recurrence coefficient $a$ for spherical harmonic translation \citep[Eq. (145)]{Zotter2009PhD}. \\ \hline
    \texttt{AmbiNav\_CoefficientB} & Returns the recurrence coefficient $b$ for spherical harmonic translation \citep[Eq. (146)]{Zotter2009PhD}. \\ \hline
    \texttt{AmbiNav\_InterpolationFilters} & Computes regularized least-squares interpolation filters as defined by \citet{TylkaChoueiri2016}. \\ \hline
    \texttt{AmbiNav\_InterpolationWeights} & Computes linear interpolation weights for specified grid points and query points. \\ \hline
    \texttt{AmbiNav\_KDThreshold} & Returns the pre-defined minimum translation distance, approximately 1 mm at 10 Hz. \\ \hline
    \texttt{AmbiNav\_Pitch90} & Returns the ambisonics rotation matrix for $90^\circ$ pitch. \\ \hline
    \texttt{AmbiNav\_PlaneWaveTranslation} & Computes plane-wave translation coefficients for a specified grid of directions and a specified translation vector. \\ \hline
    \texttt{AmbiNav\_Roll90} & Returns the ambisonics rotation matrix for $90^\circ$ roll. \\ \hline
    \texttt{AmbiNav\_Start} & Starts the AmbiNav Toolkit and the 3D3A MATLAB Toolbox. \\ \hline
    \texttt{AmbiNav\_Translation} & Computes ambisonics translation coefficients for a specified translation vector. \\ \hline
    \texttt{AmbiNav\_TriangulateSource} & Estimates the source position given microphone positions and corresponding directions-of-arrival. \\ \hline
    \texttt{AmbiNav\_Yaw90} & Returns the ambisonics rotation matrix for $90^\circ$ yaw. \\ \hline
    \texttt{AmbiNav\_YawRotation} & Returns the ambisonics rotation matrix for a specified yaw. \\ \hline
    \texttt{AmbiNav\_ZRotation} & Returns the ambisonics rotation matrix to align the $z$-axis to a given direction. \\ \hline
    \texttt{AmbiNav\_ZTranslation} & Returns the ambisonics translation coefficients for translation along the $z$-axis. \\ \hline
    \end{tabular}
    \caption{Core MATLAB functions in the toolkit.}
    \label{tab:Functions}
\end{table}

\begin{table}
\centering
  \begin{tabular}{| l | p{11cm} |}
    \hline
    \textbf{Function} & \textbf{Description} \\ \hline
    \texttt{gumerov2005} & Ambisonics navigation using translation coefficients \citep{GumerovDuraiswami2005,Zotter2009PhD}. \\ \hline
    \texttt{schultz2013} & Ambisonics navigation using plane-wave translation \citep{SchultzSpors2013}. \\ \hline
    \texttt{southern2009} & Ambisonics navigation using linear interpolation \citep{Southern2009}. \\ \hline
    \texttt{thiergart2013} & Ambisonics navigation via sound field analysis and modeling \citep{Thiergart2013}. \\ \hline
    \texttt{thiergart2013\_analysis} & Time-frequency domain analysis of a sound field \citep{Thiergart2013}. \\ \hline
    \texttt{thiergart2013\_synthesis} & Ambisonics rendering of a modeled sound field \citep{Thiergart2013}. \\ \hline
    \texttt{tylka2016} & Ambisonics navigation using least-squares interpolation filters \citep{TylkaChoueiri2016}. \\ \hline
    \end{tabular}
    \caption{Virtual navigation methods implemented in the toolkit.}
    \label{tab:Methods}
\end{table}

\section*{Acknowledgements}
This toolkit requires the 3D3A Lab's MATLAB Toolbox\footnote{Available online:~\url{https://github.com/PrincetonUniversity/3D3A-MATLAB-Toolbox}}
by R.~Sridhar and J.~G.~Tylka.

% References
\bibliographystyle{unsrtnat}
\bibliography{AmbiNav_manual}

\end{document}